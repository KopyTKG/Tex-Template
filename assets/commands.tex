
% Toto makro definuje kapitolu, která není očíslovaná, ale je uvedena v~obsahu.
\def\chapwithtoc#1{
\chapter*{#1}
\addcontentsline{toc}{chapter}{#1}
}


\DefineBibliographyStrings{czech}{ % Definování vlastního názvu pro bibliografii v češtině
  bibliography = Seznam použité literatury,
}

% TODO
\newcommand{\todo}[1]{\textcolor{red}{(\noindent TODO: #1.)}}


% Celá jména
\newcommand{\name}[1]{\mbox{\textsc{#1}}}


% Cesty
\newcommand{\appendixpath}{components/appendix}

% Images
\newcommand{\subfigwidth}{6cm}        % šířka okénka pro "podobrázky"
\newcommand{\wrappedfigwidth}{6cm}
\newcommand{\fullhd}{0.2}             % měřítko Full HD obrázku
\newcommand{\normalipe}{0.8}          % standardní měřítko IPE obrázku
\newcommand{\fractalscale}{0.3}       % měřítko obrázku softwarem vygenerovaného fraktálu

%%% Prostředí pro sazbu kódu, případně vstupu/výstupu počítačových
%%% programů. (Vyžaduje balíček fancyvrb -- fancy verbatim.)

\DefineVerbatimEnvironment{code}{Verbatim}{fontsize=\small, frame=single}

% Zapne černé "slimáky" na koncích řádků, které přetekly, abychom si
% jich lépe všimli.
\overfullrule=1mm

% Trochu volnější nastavení dělení slov, než je default.
\lefthyphenmin=2
\righthyphenmin=2

\setlength{\parskip}{0.3em}

