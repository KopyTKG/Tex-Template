


% Diameter symbol settings
\DeclareRobustCommand{\slashcirc}{{\mathpalette\doslashcirc\relax}}
\makeatletter
\newcommand{\doslashcirc}[2]{
	\sbox\z@{$#1\m@th\circ$}
	\setlength\unitlength{\wd\z@}
	\begin{picture}(1,1)
		\roundcap
		\put(0,0){\box\z@}
		\put(0,0){\line(1,1){1}}
	  \end{picture}%
}
\makeatother

% Symbol průměru
\newcommand{\diameter}[2]{$\slashcirc #1$#2}

% Matematické symboly
\newcommand{\R}{\mathbb{R}}
\newcommand{\C}{\mathbb{C}}
\newcommand{\N}{\mathbb{N}}
\newcommand{\Q}{\mathbb{Q}}
\newcommand{\Z}{\mathbb{Z}}
\newcommand{\defin}{\(=_{df}\)}
\def\doubleunderline#1{\underline{\underline{#1}}}

% Limita
\newcommand{\limit}[1]{\lim_{n\rightarrow\infty}#1}

% Derivace 1 řádu
\newcommand{\derf}[1]{#1'}

% Derivace 2 řádu
\newcommand{\ders}[1]{#1''}

% TODO
\newcommand{\todo}[1]{\textcolor{red}{(\noindent TODO: #1.)}}

% Symbol stupně
\newcommand{\degree}[1]{#1^\circ}

% Celá jména
\newcommand{\name}[1]{\mbox{\textsc{#1}}}

% Logické operátory
\renewcommand{\implies}{\Longrightarrow}
\renewcommand{\impliedby}{\Longleftarrow}
\renewcommand{\iff}{\Longleftrightarrow}

% Cesty
\newcommand{\chapterpath}[1]{components/ch#1}
\newcommand{\sectionpath}[1]{components/ch#1/sections}
\newcommand{\literaturepath}{components/literature}
\newcommand{\appendixpath}{components/appendix}

% Images
\newcommand{\subfigwidth}{6cm}        % šířka okénka pro "podobrázky"
\newcommand{\wrappedfigwidth}{6cm}
\newcommand{\fullhd}{0.2}             % měřítko Full HD obrázku
\newcommand{\normalipe}{0.8}          % standardní měřítko IPE obrázku
\newcommand{\fractalscale}{0.3}       % měřítko obrázku softwarem vygenerovaného fraktálu

%%% Prostředí pro sazbu kódu, případně vstupu/výstupu počítačových
%%% programů. (Vyžaduje balíček fancyvrb -- fancy verbatim.)

\DefineVerbatimEnvironment{code}{Verbatim}{fontsize=\small, frame=single}

%%% Makra pro definice, věty, tvrzení, příklady, ... (vyžaduje baliček amsthm)

\theoremstyle{plain}
\newtheorem{theorem}{Věta}[section]
\newtheorem{lemma}[theorem]{Lemma}
\newtheorem{proposition}[theorem]{Tvrzení}
\newtheorem{corollary}[theorem]{Důsledek}
\newtheorem*{proposition*}{Tvrzení}

\theoremstyle{definition}
\newtheorem{definition}[theorem]{Definice}
\newtheorem{example}[theorem]{Příklad}
\newtheorem{remark}[theorem]{Poznámka}
\newtheorem{convention}[theorem]{Úmluva}
\newtheorem{denoting}[theorem]{Značení}

% Zapne černé "slimáky" na koncích řádků, které přetekly, abychom si
% jich lépe všimli.
\overfullrule=1mm

% Trochu volnější nastavení dělení slov, než je default.
\lefthyphenmin=2
\righthyphenmin=2

\setlength{\parskip}{0.3em}

% Toto makro definuje kapitolu, která není očíslovaná, ale je uvedena v~obsahu.
\def\chapwithtoc#1{
\chapter*{#1}
\addcontentsline{toc}{chapter}{#1}
}