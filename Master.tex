\documentclass[12pt,a4paper]{report}

%% Vložení užitečných balíčků
% Geometry and page layout
\usepackage[left=3cm,right=3cm,top=3cm,bottom=3cm]{geometry}

% Fonts and locale
\usepackage{fontspec}
% \usepackage[czech]{babel}
\setmainfont{Geist.ttf}[
  Path = ./assets/fonts/,        % Adjust path to your font directory
  Extension = .ttf,
  UprightFont = *-Regular,
  BoldFont = *-Bold,
]


% Document enhancements
\usepackage[pdfa, colorlinks=false, urlcolor=black, bookmarksopen=true]{hyperref}
\usepackage[backend=biber, style=iso-numeric]{biblatex}
\usepackage{titlesec}

% Graphics and tables
\usepackage{graphicx, xcolor, booktabs, wrapfig}
\graphicspath{{images/}}
\usepackage{pgfplots}
\pgfplotsset{compat=1.18}
\usetikzlibrary{arrows}

% Text and math formatting
\usepackage{amsmath, bm, mathrsfs}
\usepackage{microtype, paralist, listings, textcomp, enumitem, fancyvrb, indentfirst, tocbibind, float, cancel, ragged2e, dcolumn}
\usepackage{needspace, subcaption}
\usepackage[czech]{babel}

\justifying
\lstset{upquote=true}

% Miscellaneous
\usepackage{lipsum} % For placeholder text


%% Definice různých užitečných maker (viz popis uvnitř souboru)



% Diameter symbol settings
\DeclareRobustCommand{\slashcirc}{{\mathpalette\doslashcirc\relax}}
\makeatletter
\newcommand{\doslashcirc}[2]{
	\sbox\z@{$#1\m@th\circ$}
	\setlength\unitlength{\wd\z@}
	\begin{picture}(1,1)
		\roundcap
		\put(0,0){\box\z@}
		\put(0,0){\line(1,1){1}}
	  \end{picture}%
}
\makeatother

% Symbol průměru
\newcommand{\diameter}[2]{$\slashcirc #1$#2}

% Matematické symboly
\newcommand{\R}{\mathbb{R}}
\newcommand{\C}{\mathbb{C}}
\newcommand{\N}{\mathbb{N}}
\newcommand{\Q}{\mathbb{Q}}
\newcommand{\Z}{\mathbb{Z}}
\newcommand{\defin}{\(=_{df}\)}
\def\doubleunderline#1{\underline{\underline{#1}}}

% Limita
\newcommand{\limit}[1]{\lim_{n\rightarrow\infty}#1}

% Derivace 1 řádu
\newcommand{\derf}[1]{#1'}

% Derivace 2 řádu
\newcommand{\ders}[1]{#1''}

% TODO
\newcommand{\todo}[1]{\textcolor{red}{(\noindent TODO: #1.)}}

% Symbol stupně
\newcommand{\degree}[1]{#1^\circ}

% Celá jména
\newcommand{\name}[1]{\mbox{\textsc{#1}}}

% Logické operátory
\renewcommand{\implies}{\Longrightarrow}
\renewcommand{\impliedby}{\Longleftarrow}
\renewcommand{\iff}{\Longleftrightarrow}

% Cesty
\newcommand{\chapterpath}[1]{components/ch#1}
\newcommand{\sectionpath}[1]{components/ch#1/sections}
\newcommand{\literaturepath}{components/literature}
\newcommand{\appendixpath}{components/appendix}

% Images
\newcommand{\subfigwidth}{6cm}        % šířka okénka pro "podobrázky"
\newcommand{\wrappedfigwidth}{6cm}
\newcommand{\fullhd}{0.2}             % měřítko Full HD obrázku
\newcommand{\normalipe}{0.8}          % standardní měřítko IPE obrázku
\newcommand{\fractalscale}{0.3}       % měřítko obrázku softwarem vygenerovaného fraktálu

%%% Prostředí pro sazbu kódu, případně vstupu/výstupu počítačových
%%% programů. (Vyžaduje balíček fancyvrb -- fancy verbatim.)

\DefineVerbatimEnvironment{code}{Verbatim}{fontsize=\small, frame=single}

%%% Makra pro definice, věty, tvrzení, příklady, ... (vyžaduje baliček amsthm)

\theoremstyle{plain}
\newtheorem{theorem}{Věta}[section]
\newtheorem{lemma}[theorem]{Lemma}
\newtheorem{proposition}[theorem]{Tvrzení}
\newtheorem{corollary}[theorem]{Důsledek}
\newtheorem*{proposition*}{Tvrzení}

\theoremstyle{definition}
\newtheorem{definition}[theorem]{Definice}
\newtheorem{example}[theorem]{Příklad}
\newtheorem{remark}[theorem]{Poznámka}
\newtheorem{convention}[theorem]{Úmluva}
\newtheorem{denoting}[theorem]{Značení}

% Zapne černé "slimáky" na koncích řádků, které přetekly, abychom si
% jich lépe všimli.
\overfullrule=1mm

% Trochu volnější nastavení dělení slov, než je default.
\lefthyphenmin=2
\righthyphenmin=2

\setlength{\parskip}{0.3em}

% Toto makro definuje kapitolu, která není očíslovaná, ale je uvedena v~obsahu.
\def\chapwithtoc#1{
\chapter*{#1}
\addcontentsline{toc}{chapter}{#1}
}

%%% Údaje o~práci

%%% Název práce v~jazyce práce (přesně podle zadání)
\def\NazevPrace{OLAP a ClickHouse}

%%% Typ práce
\def\TypPrace{Seminární práce}

%%% Jméno autora
\def\AutorPrace{Martin Kopecký}



%%% Rok odevzdání
\def\RokOdevzdani{2024}

% Studijní program a~obor
\def\StudijniProgram{Aplikovaná informatika}
\def\StudijniObor{Informační systémy}


%%% 3 až 5 klíčových slov (doporučeno), každé uzavřeno ve složených závorkách
\def\KlicovaSlova{%
{}, {klicova slova}
}
%\def\KlicovaSlovaEN{%
%{keywords}
%}

%%% Balíček hyperref, kterým jdou vyrábět klikací odkazy v~PDF,
%%% ale hlavně ho používáme k~uložení metadat do PDF (včetně obsahu).
%%% Většinu nastavítek přednastaví balíček pdfx.
\hypersetup{unicode}
\hypersetup{breaklinks=true}


\renewcommand{\bibname}{Seznam použité literatury}
\addbibresource{./literature/literature.bib}

%% Titulní strana a~různé povinné informační strany
\begin{document}
%%% Titulní strana práce a další povinné informační strany

%%% Titulní strana práce

\pagestyle{empty}
\hypersetup{pageanchor=false}

\begin{center}

%\centerline{\mbox{\includegraphics[width=166mm]{components/images/ujep.png}}}
\null


{\bf\Large Univerzita Jana Evangelisty Purkyně \\ v Ústí nad Labem}

\vspace{2mm}

{\bf\Large Přírodovědecká fakulta}

\vspace{20mm}

\includegraphics[width=80mm]{ujep.png}

\vspace{20mm}


{\LARGE\bfseries\NazevPrace}

{\large\bfseries\TypPrace}

\vfill


\begin{flushleft}
\begin{tabular}{>{\bfseries}r l}
    \textbf{Vypracoval:} & \AutorPrace \\
    % Vedoucí práce:\Vedouci 
    % Konzultant: \Vedouci
    \\

    \textbf{Studijní program:}  &  \StudijniProgram \\
    \textbf{Studijní obor:}  &   \StudijniObor \\
\end{tabular}
\end{flushleft}

\vfill

% Zde doplňte rok
\large{Ú\raisebox{-0.75ex}{\textsuperscript{STÍ NAD}} L\raisebox{-0.75ex}{\textsuperscript{ABEM}} \RokOdevzdani}

\end{center}

\newpage

\openright
\pagestyle{plain}
\pagenumbering{arabic}
\setcounter{page}{1}

%%% Strana s automaticky generovaným obsahem bakalářské práce

\tableofcontents

%%% Jednotlivé kapitoly práce jsou pro přehlednost uloženy v~samostatných souborech
\chapter{Test page}
Lorem ipsum dolor sit amet, consectetur adipiscing elit. Suspendisse volutpat lorem eget ligula sagittis consectetur. Aliquam libero est, ullamcorper ut vestibulum ac, efficitur non ipsum. Nulla posuere, nibh et facilisis facilisis, odio purus dictum sem, in tempus ante tellus quis ante. Nunc tortor ipsum, consequat sed sagittis sed, vehicula vitae orci. In rutrum dolor quis nulla sollicitudin, a imperdiet ipsum vestibulum. Aliquam erat volutpat. Morbi interdum bibendum arcu eget tempus. Aenean a lacus dui.

\section{Section 1}
Aliquam in urna ut tellus fringilla commodo. In hac habitasse platea dictumst. Vestibulum vitae sem id felis sagittis tincidunt id a metus. Etiam at leo arcu. Aliquam iaculis dolor a dui elementum, et aliquet nibh accumsan. Nam tristique pulvinar erat non bibendum. Aliquam tincidunt consequat velit at dignissim. Curabitur ornare, purus quis tempus pulvinar, diam elit sodales nibh, at vulputate arcu sapien sit amet eros.

\section{Section 2}
Suspendisse at neque non odio sagittis aliquet. Aliquam ac nisl imperdiet, fermentum ex nec, tristique lacus. Sed sed mi tincidunt est hendrerit auctor. Interdum et malesuada fames ac ante ipsum primis in faucibus. Integer non lectus urna. Donec tincidunt pellentesque hendrerit. Etiam placerat risus commodo, eleifend libero et, laoreet justo.

Praesent facilisis eget ipsum et dignissim. Cras vel dignissim enim. Praesent interdum, mi et scelerisque laoreet, orci diam eleifend purus, in tincidunt diam quam quis elit. Nulla ultrices erat nisi, sed porttitor erat lobortis sed. Sed tempor dapibus ultricies. Pellentesque habitant morbi tristique senectus et netus et malesuada fames ac turpis egestas. In id maximus nisl. Donec in ipsum posuere, suscipit leo id, convallis purus.

Nullam ac nibh sit amet neque elementum venenatis. Etiam eu tristique est, rhoncus congue diam. Aliquam luctus urna a purus porta, in varius nisi cursus. Quisque ipsum arcu, aliquam finibus venenatis vel, venenatis a dui. Nunc in finibus diam. Sed ex orci, sagittis ac elit id, accumsan consequat enim. In tempor ipsum lacus, id auctor ex lobortis in. Fusce fringilla congue orci a tempus. Curabitur non elit quis metus tempus eleifend. Suspendisse auctor iaculis risus, non accumsan nisl. Sed gravida nunc ut euismod blandit.


%%% Seznam použité literatury
\include{./literature/literature}

\openright
\end{document}