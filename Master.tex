\documentclass[12pt,a4paper]{report}

% Geometry and page layout
\usepackage[left=3cm,right=3cm,top=3cm,bottom=3cm]{geometry}

% Fonts and locale
\usepackage{fontspec}
% \usepackage[czech]{babel}
\setmainfont{Geist.ttf}[
  Path = ./assets/fonts/,        % Adjust path to your font directory
  Extension = .ttf,
  UprightFont = *-Regular,
  BoldFont = *-Bold,
]


% Document enhancements
\usepackage[pdfa, colorlinks=false, urlcolor=black, bookmarksopen=true]{hyperref}
\usepackage[backend=biber, style=iso-numeric]{biblatex}
\usepackage{titlesec}

% Graphics and tables
\usepackage{graphicx, xcolor, booktabs, wrapfig}
\graphicspath{{images/}}
\usepackage{pgfplots}
\pgfplotsset{compat=1.18}
\usetikzlibrary{arrows}

% Text and math formatting
\usepackage{amsmath, bm, mathrsfs}
\usepackage{microtype, paralist, listings, textcomp, enumitem, fancyvrb, indentfirst, tocbibind, float, cancel, ragged2e, dcolumn}
\usepackage{needspace, subcaption}
\usepackage[czech]{babel}

\justifying
\lstset{upquote=true}

% Miscellaneous
\usepackage{lipsum} % For placeholder text


%%% Údaje o~práci

% Název práce v~jazyce práce (přesně podle zadání)
\def\NazevPrace{Bezpočnost počítačových sítí}

% Název práce v~angličtině
\def\NazevPraceEN{\todo{Title}}

% Typ práce
\def\TypPrace{Seminární práce}

% Jméno autora
\def\AutorPrace{Martin Kopecký}



% Rok odevzdání
\def\RokOdevzdani{\todo{?}}

% Název katedry nebo ústavu, kde byla práce oficiálně zadána
% (dle Organizační struktury MFF UK, případně plný název pracoviště mimo MFF)
\def\Katedra{\todo{Katedra}}
\def\KatedraEN{\todo{Department}}

% Jedná se o~katedru (department) nebo o~ústav (institute)?
\def\TypPracoviste{Katedra}
\def\TypPracovisteEN{Department}

% Vedoucí práce: Jméno a~příjmení s~tituly
\def\Vedouci{\todo{Vedouci}}

% Pracoviště vedoucího (opět dle Organizační struktury MFF)
\def\KatedraVedouciho{\todo{Katedra}}
\def\KatedraVedoucihoEN{\todo{Department}}

% Studijní program a~obor
\def\StudijniProgram{Aplikovaná informatika}
\def\StudijniObor{Informační systémy}

% Nepovinné poděkování (vedoucímu práce, konzultantovi, tomu, kdo
% zapůjčil software, literaturu apod.)
\def\Podekovani{%
\todo{Doplnit poděkování}
\input{acknowledgements.txt}
}

% Abstrakt (doporučený rozsah cca 80-200 slov; nejedná se o~zadání práce)
\def\Abstrakt{%
\todo{Doplnit abstrakt (česky)}
\input{abstract_cz.txt}
}
\def\AbstraktEN{%
\todo{Doplnit abstrakt (anglicky)}
\input{abstract_en.txt}
}

% 3 až 5 klíčových slov (doporučeno), každé uzavřeno ve složených závorkách
\def\KlicovaSlova{%
{}, {klicova slova}
}
\def\KlicovaSlovaEN{%
{keywords}
}

%% Balíček hyperref, kterým jdou vyrábět klikací odkazy v~PDF,
%% ale hlavně ho používáme k~uložení metadat do PDF (včetně obsahu).
%% Většinu nastavítek přednastaví balíček pdfx.
\hypersetup{unicode}
\hypersetup{breaklinks=true}

%% Definice různých užitečných maker (viz popis uvnitř souboru)



% Diameter symbol settings
\DeclareRobustCommand{\slashcirc}{{\mathpalette\doslashcirc\relax}}
\makeatletter
\newcommand{\doslashcirc}[2]{
	\sbox\z@{$#1\m@th\circ$}
	\setlength\unitlength{\wd\z@}
	\begin{picture}(1,1)
		\roundcap
		\put(0,0){\box\z@}
		\put(0,0){\line(1,1){1}}
	  \end{picture}%
}
\makeatother

% Symbol průměru
\newcommand{\diameter}[2]{$\slashcirc #1$#2}

% Matematické symboly
\newcommand{\R}{\mathbb{R}}
\newcommand{\C}{\mathbb{C}}
\newcommand{\N}{\mathbb{N}}
\newcommand{\Q}{\mathbb{Q}}
\newcommand{\Z}{\mathbb{Z}}
\newcommand{\defin}{\(=_{df}\)}
\def\doubleunderline#1{\underline{\underline{#1}}}

% Limita
\newcommand{\limit}[1]{\lim_{n\rightarrow\infty}#1}

% Derivace 1 řádu
\newcommand{\derf}[1]{#1'}

% Derivace 2 řádu
\newcommand{\ders}[1]{#1''}

% TODO
\newcommand{\todo}[1]{\textcolor{red}{(\noindent TODO: #1.)}}

% Symbol stupně
\newcommand{\degree}[1]{#1^\circ}

% Celá jména
\newcommand{\name}[1]{\mbox{\textsc{#1}}}

% Logické operátory
\renewcommand{\implies}{\Longrightarrow}
\renewcommand{\impliedby}{\Longleftarrow}
\renewcommand{\iff}{\Longleftrightarrow}

% Cesty
\newcommand{\chapterpath}[1]{components/ch#1}
\newcommand{\sectionpath}[1]{components/ch#1/sections}
\newcommand{\literaturepath}{components/literature}
\newcommand{\appendixpath}{components/appendix}

% Images
\newcommand{\subfigwidth}{6cm}        % šířka okénka pro "podobrázky"
\newcommand{\wrappedfigwidth}{6cm}
\newcommand{\fullhd}{0.2}             % měřítko Full HD obrázku
\newcommand{\normalipe}{0.8}          % standardní měřítko IPE obrázku
\newcommand{\fractalscale}{0.3}       % měřítko obrázku softwarem vygenerovaného fraktálu

%%% Prostředí pro sazbu kódu, případně vstupu/výstupu počítačových
%%% programů. (Vyžaduje balíček fancyvrb -- fancy verbatim.)

\DefineVerbatimEnvironment{code}{Verbatim}{fontsize=\small, frame=single}

%%% Makra pro definice, věty, tvrzení, příklady, ... (vyžaduje baliček amsthm)

\theoremstyle{plain}
\newtheorem{theorem}{Věta}[section]
\newtheorem{lemma}[theorem]{Lemma}
\newtheorem{proposition}[theorem]{Tvrzení}
\newtheorem{corollary}[theorem]{Důsledek}
\newtheorem*{proposition*}{Tvrzení}

\theoremstyle{definition}
\newtheorem{definition}[theorem]{Definice}
\newtheorem{example}[theorem]{Příklad}
\newtheorem{remark}[theorem]{Poznámka}
\newtheorem{convention}[theorem]{Úmluva}
\newtheorem{denoting}[theorem]{Značení}

% Zapne černé "slimáky" na koncích řádků, které přetekly, abychom si
% jich lépe všimli.
\overfullrule=1mm

% Trochu volnější nastavení dělení slov, než je default.
\lefthyphenmin=2
\righthyphenmin=2

\setlength{\parskip}{0.3em}

% Toto makro definuje kapitolu, která není očíslovaná, ale je uvedena v~obsahu.
\def\chapwithtoc#1{
\chapter*{#1}
\addcontentsline{toc}{chapter}{#1}
}

%% Titulní strana a~různé povinné informační strany
\begin{document}
%%% Titulní strana práce a další povinné informační strany

%%% Titulní strana práce

\pagestyle{empty}
\hypersetup{pageanchor=false}

\begin{center}

%\centerline{\mbox{\includegraphics[width=166mm]{components/images/ujep.png}}}
\null

\vspace{20mm}

{\bf\LARGE Univerzita Jana Evangelisty Purkyně v Ústí nad Labem}

\vspace{8mm}

{\bf\Large Přírodovědecká fakulta}

\vfill

\vspace{5mm}

{\Large\bfseries\NazevPrace}

{\large\bfseries\TypPrace}

\vfill


\begin{figure}[htp]
    Vypracoval:\AutorPrace

    Vedoucí práce:\Vedouci 

    Konzultant: \Vedouci

    \vspace{5mm}

    Studijní program:\StudijniProgram
    
    Studijní obor:\StudijniObor 
\end{figure}

\vfill

% Zde doplňte rok
Ústí nad Labem \RokOdevzdani

\end{center}

%\newpage

%%% Následuje vevázaný list -- kopie podepsaného "Zadání diplomové práce".
%%% Toto zadání NENÍ součástí elektronické verze práce, nescanovat.

%%% Strana s čestným prohlášením k diplomové práci

%\openright
%\hypersetup{pageanchor=true}
%\pagestyle{plain}
%\pagenumbering{roman}
%\vglue 0pt plus 1fill

%\noindent
%Prohlašuji, že jsem tuto diplomovou práci vypracoval(a) samostatně a výhradně
%s~použitím citovaných pramenů, literatury a dalších odborných zdrojů.
%Tato práce nebyla využita k získání jiného nebo stejného titulu.
%
%\medskip\noindent
%Beru na~vědomí, že se na moji práci vztahují práva a povinnosti vyplývající
%ze zákona č. 121/2000 Sb., autorského zákona v~platném znění, zejména skutečnost,
%že Univerzita Karlova má právo na~uzavření licenční smlouvy o~užití této
%práce jako školního díla podle §60 odst. 1 autorského zákona.
%
%\vspace{10mm}
%
%\hbox{\hbox to 0.5\hsize{%
%V \hbox to 6em{\dotfill} dne \hbox to 6em{\dotfill}
%\hss}\hbox to 0.5\hsize{\dotfill\quad}}
%\smallskip
%\hbox{\hbox to 0.5\hsize{}\hbox to 0.5\hsize{\hfil Podpis autora\hfil}}
%
%\vspace{20mm}
%\newpage

%%% Poděkování

%\openright
%
%\noindent
%\Podekovani
%
%\newpage

%%% Povinná informační strana diplomové práce

%\openright
%
%\vbox to 0.5\vsize{
%\setlength\parindent{0mm}
%\setlength\parskip{5mm}
%
%Název práce:
%\NazevPrace
%
%Autor:
%\AutorPrace
%
%\TypPracoviste:
%\Katedra
%
%Vedoucí diplomové práce:
%\Vedouci, \KatedraVedouciho
%
%Abstrakt:
%\Abstrakt
%
%Klíčová slova:
%\KlicovaSlova
%
%\vss}\nobreak\vbox to 0.49\vsize{
%\setlength\parindent{0mm}
%\setlength\parskip{5mm}
%
%Title:
%\NazevPraceEN
%
%Author:
%\AutorPrace
%
%\TypPracovisteEN:
%\KatedraEN
%
%Supervisor:
%\Vedouci, \KatedraVedoucihoEN
%
%Abstract:
%\AbstraktEN
%
%Keywords:
%\KlicovaSlovaEN
%
%\vss}

\newpage

\openright
\pagestyle{plain}
\pagenumbering{arabic}
\setcounter{page}{1}

%%% Strana s automaticky generovaným obsahem bakalářské práce

\tableofcontents

%%% Jednotlivé kapitoly práce jsou pro přehlednost uloženy v~samostatných souborech
\include{\chapterpath{00}/ch00_predmluva.tex}
%\include{\chapterpath{01}/ch01_uvod_do_fraktalu.tex}

%%% Seznam použité literatury
%\include{\literaturepath/literature.tex}

%%% Obrázky v~bakalářské práci
%%% (pokud jich je malé množství, obvykle není třeba seznam uvádět)
\listoffigures

%%% Tabulky v~bakalářské práci (opět nemusí být nutné uvádět)
%%% U matematických prací může být lepší přemístit seznam tabulek na začátek práce.
% \listoftables

%%% Přílohy k~bakalářské práci, existují-li. Každá příloha musí být alespoň jednou
%%% odkazována z vlastního textu práce. Přílohy se číslují.
%%%
%%% Do tištěné verze se spíše hodí přílohy, které lze číst a~prohlížet (dodatečné
%%% tabulky a~grafy, různé textové doplňky, ukázky výstupů z počítačových programů,
%%% apod.). Do elektronické verze se hodí přílohy, které budou spíše používány
%%% v~elektronické podobě než čteny (zdrojové kódy programů, datové soubory,
%%% interaktivní grafy apod.). Elektronické přílohy se nahrávají do SISu a~lze
%%% je také do práce vložit na CD/DVD. Povolené formáty souborů specifikuje
%%% opatření rektora č. 72/2017.

\appendix

\openright
\end{document}